\documentclass[utf8]{article}

\usepackage[utf8]{inputenc}

\usepackage[parfill]{parskip}

\usepackage{amsmath}
\usepackage{amssymb}
\usepackage{amsfonts}
\usepackage{graphicx}
\usepackage{float}
\usepackage{listingsutf8}
\usepackage{hyperref}
\usepackage[dvipsnames]{xcolor}

\usepackage{fullpage}

%------------------------------------------------------

\usepackage{listings}
\usepackage{xcolor}

\definecolor{color0}{RGB}{147, 147, 147} %<--- I've changed this to make it more visible
\definecolor{color1}{RGB}{186, 033, 033}
\definecolor{color2}{RGB}{000, 128, 000}
\definecolor{color3}{RGB}{064, 128, 128}
\definecolor{color4}{RGB}{170, 034, 255}

\lstdefinelanguage{clips}{
  mathescape = true,
  sensitive        = true,
  morecomment      = [l]{;},
  showstringspaces = false,
  morestring       = [b]",
}

% egreg's modulo macro (see https://tex.stackexchange.com/a/34449/21891)
\def\truncdiv#1#2{((#1-(#2-1)/2)/#2)}
\def\moduloop#1#2{(#1-\truncdiv{#1}{#2}*#2)}
\def\modulo#1#2{\number\numexpr\moduloop{#1}{#2}\relax}


\makeatletter

% a TeX counter to keep track of the nesting level
\newcount\netParensCount@clisp

% Modify how ( and ) get typeset depending on the value of the counter
% (Based on Ulrike Fischer's approach to modifying characters in listings;
% see https://tex.stackexchange.com/a/231927/21891)
\lst@CCPutMacro
\lst@ProcessOther{`(}{{%
  \ifnum\lst@mode=\lst@Pmode\relax%
    \rainbow@clisp{(}%
    \global\advance\netParensCount@clisp by \@ne%
  \else
    (%
  \fi
}}%
\lst@ProcessOther{`)}{{%
  \ifnum\lst@mode=\lst@Pmode\relax%
    \global\advance\netParensCount@clisp by \m@ne%
    \rainbow@clisp{)}%
  \else
    )%
  \fi
}}%
\@empty\z@\@empty

% Color its argument based on the value of the \netParensCount@clisp counter
% (modulo 5)
\newcommand\rainbow@clisp[1]{%
  \ifcase\modulo\netParensCount@clisp 5\relax%
    \textcolor{color0}{#1}%
  \or
    \textcolor{color1}{#1}%
  \or
    \textcolor{color2}{#1}%
  \or
    \textcolor{color3}{#1}%
  \else
    \textcolor{color4}{#1}%
  \fi
}

\lst@AddToHook{PreInit}{%
  \global\netParensCount@clisp 0\relax%
}

\makeatother

\lstnewenvironment{clips-code}
  {\lstset{language=clips}}
  {}


% -----------------------------------------------------


\title{SAT-solitaire}
\author{Fonseca Loïc - Bourgeois Noé}
\date{November 2021}

\begin{document}
\maketitle
\tableofcontents

\newpage

% -----------------------------------------------------

\section{Introduction}
Le Solitaire est un jeu de plateau classique à un seul joueur. 
Le plateau de jeu comporte des trous dans lesquels on peut mettre des billes. 
Pour jouer un coup au Solitaire, il faut choisir trois cases adjacentes, qui sont
toutes les trois sur une même ligne ou une même colonne du plateau. De surcroît,
exactement deux billes doivent être posées sur les trois cases, dont une posée sur
la case du milieu. On fait ”sauter” la bille qui se trouve sur une des extrémités
par dessus la bille du milieu, pour la faire atterrir dans le trou qui se trouve à
l’autre extrémité. On enlève alors la bille du milieu du plateau. La Figure 2 est
une représentation des différents coups possible, en utilisant la correspondance
entre matrice et configurations du jeu. Nous pouvons représenter une configuration du jeu par un entier positif n, et
une matrice de taille n × n dont les éléments appartiennent à {-1, 0, 1}. Si la
valeur de la case (i, j) de la matrice est égale à -1, alors il n’y a pas de trous
sur le plateau à la position (i, j); si cette valeur est égale à 0, alors il y a un
trou sur le plateau `a la position (i, j), mais il n’y a pas de bille posée dessus; 
si cette valeur est égale à 1, alors il y a un trou sur le plateau à la position (i, j) avec une bille posée dessus.


\section{Formule Normale Conjonctive}
Soit n un entier, \\
M et M' deux matrices de tailles n×n 
dont les éléments sont dans \{-1, 0, 1\}. 


satisfaisable si et seulement s’il est possible de passer de
\\la configuration du Solitaire
qui correspond à M 
\\à 
\\celle qui correspond à M'
, en effectuant un ou
plusieurs coups. 
\subsection{Composition}

\subsubsection{Variables  Booléennes}

On commence par définir les variables de nos formules. 
\\La variable x$_i,_j,_k$
\\vaut “Vrai” si le nombre k est à la ligne i, colonne j.

\subsubsection{Exactement une valeur par case}
On note que la nature du problème impose des contraintes qui ne sont pas notées dans l’énoncé. 
\\Par exemple, dans chaque case (i, j), il doit y avoir exactement une seule valeur k.

En d’autre termes pour tout i,
pour tout j, pour tout k != k'
, on n’a pas $x_{i,j,k} ~ \land ~ x_{i,j,{k^\prime}}$.
\newline
$\overset{N}{\underset{i=1}{\bigwedge}}
~~\overset{N}{\underset{j=1}{\bigwedge}}
(
~~\underset{-1<={k,k'}<=1}{\bigwedge}
\neg x_{i,j,k} ~ \lor ~ \neg x_{i,j,{k^\prime}})
 ~ \land ~ (x_{i,j,-1} ~ \lor ~ x_{i,j,0} ~ \lor ~ x_{i,j,1})$
\subsubsection{Coup}

\paragraph{Voisin}
On commence par définir ce qu’est un voisin : il s’agit d’une autre case dont les coordonnées ne sont pas éloignées de plus de 1.

Avec L le nombre de lignes de M
\\et C le nombre de colonnes de M,

$V (i, j) = {(i', j') \in [1, N]^2| max(|i - i'|, |j - j'| = 1)}$
 
 
\subparagraph{sur même ligne}
Considérant une case (i,j) 
alors une voisine se trouve en (i,j-1) 
et l'autre en (i,j+1) 

Considérant une case x$_{i,j,k}$, alors une voisine se trouve en x$_{i,j-1,k^\prime}$ et l'autre en x$_{i,j+1,k^{\prime\prime}}$

$(x_{i,j-1,k} \lor ~~ x_{i,j-1,k^\prime} )
\land ~~
x_i,_j,_k 
\land ~~
(x_{i,j+1,k^\prime} \lor ~~ x_{i,j+1,k})$


\subparagraph{sur même colonne}
Considérant une case (i,j) 
alors une voisine se trouve en (i-1,j) 
et l'autre en (i+1,j) 

variable x$_{i,j,k}$, 
alors une voisine se trouve en x$_{i-1,j,k^\prime}$ et l'autre en x$_{i+1,j,k^{\prime\prime}}$

$(x_{i-1,j,1} \lor ~~ x_{i-1,j,0} )
\land ~~
x_{i,j,1}
\land ~~
(x_{i+1,j,0} \lor ~~ x_{i+1,j,1})$

\subparagraph{non-case}
Considérant une case (i,j)
variable 
x$_{i,j,k}$, 
les cases aux positions 
(i,j-1) variable x$_{i,j-1,k}$ 
et (i,j+1) variable x$_{i,j+1,k^\prime}$, 
sont des voisines
si et seulement si k,$k^ {\prime}$ != -1

\subparagraph{Jouer un coup}
Si on peut jouer un coup, alors, 
considérant une bille à la position (i,j) x$_{i,j,k}$, 
une de ses voisines
a forcément k = 1 et l'autre k = 0.\newline

Soit si x$_{i,j-1,1}$, x$_{i,j,1}$ et x$_{i,j+1,0}$, alors on peut jouer un coup tel que x$_{i,j-1,0}$, x$_{i,j,0}$ et x$_{i,j+1,1}$
\newline

\begin{clips-code}
    ((($x_{i,j-1,1} \land \neg x_{i,j+1,1}$) $\lor$ ($ x_{i,j-1,0} \land \neg x_{i,j+1,0}$)) $\land$ (($x_{i-1,j,1} \land \neg x_{i+1,j,1}$) $\lor$ ($ x_{i-1,j,0} \land \negx_{i+1,j,0}$)))
    
    $\land$ ($x_{i,j,1}$) 

    $\land$ $\neg$($ x_{i,j-1,-1} \lor x_{i-1,j,-1} \lor x_{i,j,-1} \lor x_{i,j+1,-1} \lor x_{i+1,j,-1}$)


\end{clips-code}

% -----------------------------------------------------
\newpage

\section{Programs}

\subsection{ Mirroring Cannibal Sheeps }

\subsubsection{function}

\begin{figure}[H]
\begin{minipage}{\textwidth}
  \centering	
	\begin{lstlisting}
	function solution(m1, m2) 
    \end{lstlisting}
  \label{fig:code_exemple}
\end{minipage}
\end{figure}




\subsection{ Sheeps Arguing with Tornado }
En se basant sur le programme précédent, 
un programme qui pour toute configuration de la Figure 1 (sauf la 6 et la
4), \\
teste s’il est possible d’atteindre une configuration du même plateau
mais qui contient une seule bille là où au départ il n’y avait pas de bille.\\


\subsubsection{function}

\begin{figure}[H]
\begin{minipage}{\textwidth}
  \centering	
	\begin{lstlisting}
	function tornado(m1, m2) 
    \end{lstlisting}
  \label{fig:code_exemple}
\end{minipage}
\end{figure}

\paragraph{paragraph}:
    paragraph
    
  

\section{Résultats}
\begin{center}
\begin{tabular}{|c|c|c|c|
                |c|c|c|c|}
\hline
plateau taille & quantité billes & Mirror & Tornado \\
\hline
              &                &       &\\
             &                &               & \\
             &                &               & \\
             &                &               & \\
\hline
             &                &              & \\
\hline
\end{tabular}
\end{center}



\begin{figure}[H]
  \centering
	\includegraphics[scale=0.8]{img/Figure_1.png}
  \label{fig:logo}
\end{figure}

\newpage

\section{Discussion}
    

\section{Conclusion}


\end{document}
