\documentclass[utf8]{article}

\usepackage[utf8]{inputenc}

\usepackage[parfill]{parskip}

\usepackage{amsmath}
\usepackage{amssymb}
\usepackage{amsfonts}
\usepackage{graphicx}
\usepackage{float}
\usepackage{listingsutf8}
\usepackage{hyperref}
\usepackage[dvipsnames]{xcolor}

\usepackage{fullpage}

% -----------------------------------------------------


\title{SAT-solitaire}
\author{Fonseca Loïc - Bourgeois Noé}
\date{November 2021}

\begin{document}
\maketitle
\tableofcontents

\newpage

% -----------------------------------------------------

\section{Introduction}
Le Solitaire est un jeu de plateau classique à un seul joueur. 
Le plateau de jeu comporte des trous dans lesquels on peut mettre des billes. 
Pour jouer un coup au Solitaire, il faut choisir trois cases adjacentes, qui sont
toutes les trois sur une mˆeme ligne ou une mˆeme colonne du plateau. De surcoˆıt,
exactement deux billes doivent ˆetre pos´ees sur les trois cases, dont une pos´ee sur
la case du milieu. On fait ”sauter” la bille qui se trouve sur une des extr´emit´es
par dessus la bille du milieu, pour la faire atterrir dans le trou qui se trouve `a
l’autre extr´emit´e. On enl`eve alors la bille du milieu du plateau. La Figure 2 est
une repr´esentation des diff´erents coups possible, en utilisant la correspondance
entre matrice et configurations du jeu.Nous pouvons représenter une configuration du jeu par un entier positif n, et
une matrice de taille n × n dont les éléments appartiennent à {-1, 0, 1}. Si la
valeur de la case (i, j) de la matrice est égale à -1, alors il n’y a pas de trous
sur le plateau à la position (i, j); si cette valeur est égale à 0, alors il y a un
trou sur le plateau `a la position (i, j), mais il n’y a pas de bille posée dessus; 
si cette valeur est égale à 1, alors il y a un trou sur le plateau à la position (i, j) avec une bille posée dessus.


\section{Formule Normale Conjonctive}
Soit n un entier, \\
M et M' deux matrices de tailles n×n 
dont les éléments sont dans \{-1, 0, 1\}. 

satisfaisable si et seulement s’il est possible de passer de\\
la configuration du Solitaire
qui correspond à M \\
à \\
celle qui correspond à M'\\
, en effectuant un ou
plusieurs coups. 



\subsection{choix de variables Booléennes}


\subsection{décomposition}
\subsubsection{partie 1}




% -----------------------------------------------------
\newpage

\section{Programs}

\subsection{ Mirroring Cannibal Sheeps }

\subsubsection{function}

\begin{figure}[H]
\begin{minipage}{\textwidth}
  \centering	
	\begin{lstlisting}
	function solution(m1, m2) 
    \end{lstlisting}
  \label{fig:code_exemple}
\end{minipage}
\end{figure}




\subsection{ Sheeps Arguing with Tornado }
En se basant sur le programme précédent, 
un programme qui pour toute configuration de la Figure 1 (sauf la 6 et la
4), \\
teste s’il est possible d’atteindre une configuration du même plateau
mais qui contient une seule bille là où au départ il n’y avait pas de bille.\\


\subsubsection{function}

\begin{figure}[H]
\begin{minipage}{\textwidth}
  \centering	
	\begin{lstlisting}
	function tornado(m1, m2) 
    \end{lstlisting}
  \label{fig:code_exemple}
\end{minipage}
\end{figure}

\paragraph{paragraph}:
    paragraph
    
  

\section{Résultats}
\begin{center}
\begin{tabular}{|c|c|c|c|
                |c|c|c|c|}
\hline
plateau taille & quantité billes & Mirror & Tornado \\
\hline
              &                &       &\\
             &                &               & \\
             &                &               & \\
             &                &               & \\
\hline
             &                &              & \\
\hline
\end{tabular}
\end{center}



\begin{figure}[H]
  \centering
	\includegraphics[scale=0.8]{img/Figure_1.png}
  \label{fig:logo}
\end{figure}

\newpage

\section{Discussion}
    

\section{Conclusion}


\end{document}
